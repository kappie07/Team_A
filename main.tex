\documentclass{article}
\usepackage[utf8]{inputenc}

\title{Project proposal Team-A\\ Simulation and Modelling in Astrophysics\\ Project name}
\author{Jaro Molenkamp \& Kasper Roewen }
\date{September 2020}

\begin{document}

\maketitle

\section{Summary}%both
Here we describe the goals and methods for our project for the course Simulation and Modelling in Astrophysics (SMA)

\section{Research question}%Jaro
The goal of the project is to study the capture of binary stars in two merging galaxies, where one small galaxy spirals into the larger galaxy. The question that we seek to answer therefore is; can we find binary star systems as a result of a galaxy merger with gravitational only simulations?

\section{Method}%Kasper
We achieve our goals by setting up a galaxy as a N-body problem with multiple components (DM-halo, the stellar disk and stellar bulge). The problem will only consist of gravitational code and the mass evolution can be checked using several initial conditions:
\begin{itemize}
    \item Initial distance
    \item Initial relative velocity
    \item total mass, as well as mass ratio's
\end{itemize}

We will define a criterion for mass transfer from the small galaxy to the larger galaxy (might consider mass that scatters away). 

\section{Previous work \& importance}%Kasper importance, both previous work

%Previous work?: https://www.science.gov/topicpages/g/galaxy+merger+simulations 
%something about the capturing of stars to form a binary.

%https://pdfs.semanticscholar.org/409b/8b1ff71f862669f4c0d431ffada428c3a34c.pdf

\section{Mitigation strategy}%Jaro
With a project there is always a presence of risk. However, we strive to keep these risks to a minimum. In this project we will assume risk, try to avoid risk, control risk and monitor risk. Since we work on programming, bugs and calculation times form a risk. We will try to avoid these risks by assuming they will happen, we can therefore adapt our methods. For example, to minimize bugs we will slowly progress towards our goal, we make use of sub-questions, how does a normal galaxy merger take place (without looking at binary star capture)? Furthermore, since calculation time is most likely going to be a problem we are going to monitor the progress along the way. For example, if we look at the sub-question, how much calculation time was needed for this? Is this a reasonable amount or do we have to switch our calculation method. By doing this we control the risk at all times


{\bf SPZ: Looks good so far.Give it an appropriate title and acronym. Right sort of structure. Literature is requires, but not too many refs. Please use bibtex for the references.}
\end{document}
